\documentclass[aspectratio=1610]{beamer}
\usepackage[T1]{fontenc}
\usetheme{wildcat}


\title{How to use LaTex}
\date{February 2025}
\author{ (Shafil Jayed, Pranaav Venkatasubramanian, Jerry Steve Jeanty, Bryan Belizaire )}

% You change the titlegraphic to whatever you want, or comment it out to remove it.
\titlegraphic{\includegraphics[scale=0.25]{logo-northwestern.pdf}}

% % You can directly change the colors using the following macros.
% % You must redefine colors AFTER the theme is loaded.
% % For example, these provide shades of Yale Blue (#00356b)
% \definecolor{wcprimary}{RGB}{0,53,107}      % Main color
% \definecolor{wcprimary140}{RGB}{0, 34, 70}
% \definecolor{wcprimary130}{RGB}{0, 40, 80}
% \definecolor{wcprimary120}{RGB}{0, 45, 91}
% \definecolor{wcprimary110}{RGB}{0, 50, 102}
% \definecolor{wcprimary40}{RGB}{153, 174, 196}
% \definecolor{wcprimary30}{RGB}{179, 194, 211}
% \definecolor{wcprimary20}{RGB}{204, 215, 225}
% \definecolor{wcprimary10}{RGB}{230, 235, 240}

% % Now for the alerted orange (#bd5319) and example green (#5f712d)
% \definecolor{wcalerted}{RGB}{189,83,25}
% \definecolor{wcexample}{RGB}{95,113,45}

% % If you want to change the slide background color, 
% % you can use the following command:
%\setbeamercolor{background canvas}{bg=nupurple10!30}

% % Turn off section slides
% \AtBeginSection{}

% Change the font theme
%\usefonttheme{wildcat-overleaf}

% Change the bg pattern manually: Simple Single Color
% \renewcommand{\bgpattern}{
%     \draw[color=wcprimary,fill=wcprimary] (0,0) rectangle (\paperwidth,\paperheight);
% }



\begin{document}

\begin{frame} [How To use Latex]
\titlepage
\end{frame}

% \begin{frame}{Table of Contents}
%     \tableofcontents
% \end{frame}

\begin{frame}{Introduction}
    LaTex is a document typsetting format that is used to create documents and presentations. LaTex is able to create high-quality documents that are considered professional. 
    \\ ~ \\
    LaTex is known for its precision and how each detail is in the hands of the author. One of the best parts of LaTex is its ability to recreate and type complex mathematical symbols that are not available in most word processing programs. 
\end{frame}

\begin{frame}
    \frametitle{Install}
        LaTex is available on Mac, Linux, and Windows. To install on Windows will also require the installation MikeTex, Ghostscript, Ghostview, and WinEdit. For Mac you only need to install the the Mac version of LaTex MacTex. To install for Linux you only need to install Live Tex.
        


\end{frame}

\begin{frame}{Basic Syntax}
    All LaTex documents have a skeleton which acts as an outline for the document that is filled in.  A basic skeleton looks like the following: 
    \\ ~ \\
\textbackslash{}documentclass\{article\}  - Decides your document class including font size and spacing
    \\ ~ \\
\textbackslash{}begin\{document\} - Is where the content of the document begins
    \\ ~ \\
Hello, World! - The content of the docuement
    \\ ~ \\
\textbackslash{}end\{document\} - Is where the document ends
    
\end{frame}

\begin{frame}[fragile]
    \frametitle{More Syntax)}
    List: 
    \\
    \textbackslash{}begin\{itemize\} or \textbackslash{}begin\{enumerate\} (for numbered Lists)
    \\
    \textbackslash{}item First item
    \\
    \textbackslash{}item Second item
    \\
    \textbackslash{}end\{itemize\} or \textbackslash{}end\{enumerate\}
        \\ ~ \\
    Sections:
    \\
    \textbackslash{}section\{Introduction\}
    \\
    \textbackslash{}subsection\{Overview\}
           \\ ~ \\
    \textbackslash{}textbf\{bold\}- to bolden text
    \\
    \textbackslash{}textit\{italic\}- to italicize text
    \\
    \textbackslash{}underline\{underline\}- to underline text
\end{frame}

\begin{frame}[fragile]
    \frametitle{Math in Latex}
    Basic math equation in Latex is formatted like this:
    \\
\begin{verbatim}
\[a^2 +b^2 =c^2\] 
\end{verbatim}
\\
When written it will appear as \[a^2 +b^2 =c^2\]    
    \\
    But if you want to your equation in a sentence you will write it like this:
    \\
\begin{verbatim}
$a^2 + b^2 = c^2$
\end{verbatim}    
\\
This will then produce $a^2+b^2=c^2$
\end{frame}

\begin{frame}[fragile]
    \frametitle{More Math Syntax}
    Fractions are written: 
    \\
    \begin{verbatim}
\frac for example \frac{2}{7} would produce:

\end{verbatim}
\\
$\frac{2}{7}$
\\ - \\
Subscripts are written:
\\
\begin{verbatim}
x_3 which would produce:
\end{verbatim}
\\
$x_3$
\\ \\
Square roots are written: 
\begin{verbatim}
\sqrt{4} which would produce

\end{verbatim}
  \\
  $\sqrt{4}$

  \\
  In Latex you can also write Greek alphabets:
    \begin{verbatim}
To type Omega you would write \Omega which would produce:
\end{verbatim}
\\
$\Omega$
\\

\end{frame}

\begin{frame}[fragile]
    \frametitle{How to create a table}
      \begin{verbatim}
\begin{tabular}{|c|c|c|} - start and # of columns
\hline -  row
MON & TUES & WED \\ - contents
\hline
Jack & Jill & Bill \\
1 & 2 & 3 \\
\hline
\end{tabular} - end
      \end{verbatim}
    \begin{verbatim}

\end{verbatim}
   \begin{tabular}{|c|c|c|}
\hline
MON & TUES & WED \\
\hline
Jack & Jill & Bill \\
1 & 2 & 3 \\
\hline
\end{tabular}

\end{frame}

\begin{frame}[fragile]
    \frametitle{Citation}
    This is how to make citations in LaTex:
    \\
    \begin{verbatim}
  use BibTex a tool to help you create citations in 
  LaTex.
  \bibitem{Jackson9}
  Jack Jackson,
  \emph{A Book}.
  Old Westbury, New York,
  2nd Edition,
  2025.
  would look like this:
    \end{verbatim}
    \\
\bibitem{Jackson9}
  Jack Jackson,
  \emph{A Book}.
  Old Westbury, New York,
  2nd Edition,
  2025.
\end{frame}

\begin{frame}[fragile]
    \frametitle{Change Language}
To change the language in Latex you must do the following:
\begin{verbatim}
\usepackage[spanish]{babel}
\begin{document}
¡Hola a todos!
\end{document}
\end{verbatim}
This would create a new document that would change the language using the babel package.

\end{frame}






\section{The End}



\end{document}
